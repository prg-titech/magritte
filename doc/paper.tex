\documentclass[format=sigconf, review=true, screen=true]{acmart}

% don't print the conference ref for now
\settopmatter{printacmref=false}

\usepackage{comment}
\usepackage{mathtools}
\usepackage{amssymb}
\usepackage{amsmath}
\usepackage{graphics,graphicx}
\usepackage{pstricks,pst-node,pst-tree}
\usepackage{auto-pst-pdf}


\title{A Concurrent Semantic for Shell-Like Processes}
\author{Jeanine Adkisson}
\affiliation{Tokyo Institute of Technology}
\email{jneen@prg.is.titech.edu}

\author{Hidehiko Masuhara}
\affiliation{Tokyo Institute of Technology}
\email{masuhara@prg.is.titech.edu}

\begin{document}

\begin{abstract}
Despite the widespread use of GUI tools for configuration tasks, CLI tools, and more specifically shell-based interfaces remain a powerful standard for software distribution and integration. Several attempts have been made at solving the desktop-integration problem, most notably TCL(TODO: cite!) and Guile Scheme(TODO: cite!), but neither has been as succesful as they intended to be: the \emph{lingua franca} for a large portion of tools remains shell-oriented, relying on such concepts as standard I/O, string-based argument vectors, and the expectation that users will be able to freely pipe streams of bytes in a shell-like way.

Unfortunately, shell languages have significant known interoperability issues, such as lack of any kind of usable encapsulation, and the lack of in-language data structures to represent any kind of value other than a string. We believe that a shell-compatible language with well-defined semantics that could also function as a general-purpose systems language could go a long way towards solving the desktop-integration problem.

In this paper, we explore a potential concurrency model for such a language, inspired by such systems as Go, Clojure, and Erlang. A persistent problem with these battle-tested systems is that they don't account for closing channels, and require manual cleanup of processes. In a systems programming context, errors are commonplace and expected. We propose a channel-based communication system that integrates with a workflow compensation system and is capable of recovering from errors and, importantly, cleaning up processes when communication halts.
\end{abstract}
\maketitle




\end{document}
